\documentclass[10pt]{article}

% Required packages
\usepackage{bm,bbm}
\usepackage{amsmath,amssymb,amsthm,cancel}
\usepackage{algorithm, algpseudocode}
\usepackage{minted, caption}



% Color references
\usepackage[
    colorlinks=true, citecolor=green, linkcolor=blue]{hyperref}

\newcommand{\homework}[2]{
	\noindent
    \begin{center}
    	\framebox{
        	\vbox{
            	% Course Title and Date
            	\hbox to \hsize { \textsc{ORIE 7390 - Special Topics in
				Mathematical Programming} \hfill \textsc{#1} }
                \vspace{4mm}
                % Title of handout/homework
                \hbox to \hsize { {\Large \hfill \textsc{#2} \hfill} }
                \vspace{1mm}
            }
        }
    \end{center}
}

\newenvironment{alglist}{\begin{list}{}{\setlength{\leftmargin}{1.5cm}
\setlength{\rightmargin}{0cm}\setlength{\itemsep}{1ex}\setlength{\parsep}{1ex}}}{\end{list}}

\newcommand{\problem}[3]
{\fbox{\parbox{6in}{{\bf #1}\begin{itemize}\item{\bf Input:} {#2} \item{\bf Goal:} {#3}\end{itemize}}}}

\usepackage{latex-macros}
\usepackage{todonotes}
\usepackage{tikz,pgfplots}
\usepackage{algorithm, algpseudocode}
\usepackage[margin=1in]{geometry}
\usetikzlibrary{shapes.geometric}
%\usepackage{parskip}
\usepackage[capitalize]{cleveref}


% Use tcolorbox, define custom "Exercise" environ
\usepackage{tcolorbox}
\tcbuselibrary{theorems}

% Create Exercise environment
\newtcbtheorem[number within=section]{exercise}{Exercise}%
{}{exc}
\newtcbtheorem[number within=section]{ctheorem}{Theorem}%
{colframe=gray!30!black}{thm}
\newtcbtheorem[number within=section]{cdefinition}{Definition}%
{colframe=gray!30!black}{defn}
\newtcbtheorem[number within=section]{cproposition}{Proposition}%
{colframe=gray!30!black}{prop}

\usepackage{todonotes}

\newcommand{\fsubsection}[1]{\subsection{\textsf{#1}}}
\newcommand{\fsubsubsection}[1]{\subsubsection{\textsf{#1}}}

\newcommand{\regdiff}{\hat{\partial}}
\newcommand{\bd}[1]{\mathrm{bd}\left( #1 \right)}

\begin{document}

\allowdisplaybreaks
\everymath{\displaystyle}

\homework{\today}{Lecture Notes - Fall 2018}
\tableofcontents
\listoftodos

\section{Introduction}

\subsection{Recap}
We briefly review some fundamentals from convex analysis. Recall that, for
\[
	f : \Hbb \to \Rbb \cup \set{+\infty},
\]
where $f$ is convex and $\Hbb$ is a Hilbert space (e.g. $\Rbb^n$), we write
\[
	y \in \partial f(x) \Leftrightarrow x \in \argmin f(\cdot) - \ip{y, \cdot}.
\]
The subgradient $\partial f(x)$ is closed and convex, so it has a unique
shortest element when it is nonempty. This is because the shortest element is
given by the following convex program:
\[
	\inf_{z \in \partial f(x)} \norm{z}^2.
\]
$\norm{~}^2$ is a strongly convex function which is minimized over a closed,
convex set, hence it admits a unique minimizer. Let us call this minimizer
$\partial^o f(x)$.

It can also be shown that the direction of steepest descent is given by
$-\partial^o f(x)$. Hence, for minimization, it is natural to follow the
trajectory given by
\begin{align}
	\begin{cases}
		\dot{X}(t) = -\partial^o f(X(t)), & t \geq 0 \\
		X(0) = x_0. &
	\end{cases}
	\label{eq:descent_trajectory}
\end{align}
This raises a fundamental question:

\paragraph{Question 1: Can we really follow this trajectory?}

The following Theorem provides an affirmative answer:

\begin{ctheorem}{\cite{Brezis73}}
	There exists a trajectory defined by~\cref{eq:descent_trajectory}.
	Moreover, by~\cite{Bruck75}, if that trajectory is bounded, it converges
	weakly to a minimizer of $f$.
\end{ctheorem}

However, in order to be tractable, we would also like to control the length of
those trajectories.

\paragraph{Question 2: How do we control them?}

\subsection{History}
For smooth $f: \Rbb^n \to \Rbb$ (nonconvex):
\begin{itemize}
	\item gradient descent trajectory:
		\[
			\begin{cases}
				\dot{X}(t) = -\grad f(X(t)), & t \geq 0 \\
				X(0) = x_0 \\
			\end{cases}
		\]
		might not have finite length, even if it stays in bounded regions. The
		canonical counterexample involves a spiral ridge or a so-called
		``mexican hat''.
	\item if $f$ is analytic, we have a positive result:
		\begin{ctheorem}{{\L}ojasiewicz, 1984}
			When $f$ is real-analytic (i.e. admits a power series expansion),
			bounded gradient descent trajectories have finite length.
		\end{ctheorem}
\end{itemize}
A key tool in the Theorem presented above is the {\L}ojasiewicz inequality,
presented below:
\begin{ctheorem}{\cite{Loja59}}
	For real-analytic $f: \Rbb^n \mapsto \Rbb$, around a local minimizer
	$\hat{x}$, it holds that
	\[
		\norm{\grad f(\cdot)} \geq c \abs{f(\cdot) - f(\hat{x})}, \;
		c > 0, \; 0 \leq \mu < 1.
	\]
\end{ctheorem}
An example is $f(x) = x^2$, which satisfies the {\L}ojasiewicz inequality
around $0$ with exponent $\mu = 1/2$.


\section{Some Convex Analysis}

\subsection{Fundamentals}
The setup: $\Hbb$ is a real Hilbert space (e.g. $\Rbb^n$ for us, but could be
infinite-dimensional).
We denote $\overline{\Rbb} = \Rbb \cup \set{\pm \infty}$. For $f: \Hbb \to
\overline{\Rbb}$, we write
\begin{equation}
    \epi f = \set{(x, r) \in \Hbb \times \Rbb \mmid r \geq f(x)}
    \label{eq:epif}
\end{equation}

It holds that $f$ convex $\Leftrightarrow$ $\epi f$ convex, which is a simple
consequence of convexity. Additionally, $f$ closed implies that $\epi f$ is
closed, which is equivalent to $f$ being lower semicontinuous.

We call $f$ \textit{proper} if it is never $-\infty$, and not always $+\infty$.
Moreover, we define
\begin{equation}
    \dom f = \set{x \in \Hbb \mmid f(x) < \infty} \label{eq:domf}
\end{equation}

\begin{cdefinition}{Subgradient}{subgradient}
    For a convex function $f$, we write $y \in \partial f(x)$ meaning that
    \[
        f(z) \geq f(x) + \ip{y, z - x}, \; \forall z.
    \]
    The set $\partial f(x)$ is always a closed convex set. If it is nonempty,
    it has a \textit{unique} shortest element $\partial^{\circ} f(x)$.
\end{cdefinition}


\begin{ctheorem}{Continuity of convex functions}{conv-cont}
    For a proper convex $f: \Hbb \to \overline{\Rbb}$, at a point $x \in
    \intr{\dom f}$, we know that $f$ is locally Lipschitz on a neighbourhood
    around $x$ with constant $L$. Moreover, its subdifferential satisfies
    \[
        \partial f(x) \neq \emptyset, \quad \norm{\partial f(x)} \leq L.
    \]
\end{ctheorem}
\begin{proof}
    Via the separating hyperplane theorem (Hahn-Banach).
\end{proof}

\subsection{The Fenchel conjugate}
Recall that, for any $f: \Hbb \to \overline{\Rbb}$, we define its
\textbf{conjugate} $f^*$ by
\begin{equation}
    f^*(y) = \sup_{x} \set{\ip{x, y} - f(x)}
    \label{eq:fenchel_conjugate}
\end{equation}

Additionally, from the above definition, we deduce the so-called
\textit{Fenchel-Young inequality}:
\[
    f(x) + f^*(y) \geq \ip{x, y}, \; \forall x, y.
\]

\begin{ctheorem}{Biconjugate}{biconjugate}
    For closed, convex, proper $f: \Hbb \to \overline{\Rbb}$, it holds that
    $f^{**} = f$.
\end{ctheorem}

We briefly give some examples:
\begin{itemize}
\item Consider $\cC \subseteq \Hbb$. We define its \textit{indicator function}
    $\delta_{\cC}(x) = \begin{cases}
        0, & x \in \cC \\
        +\infty, & x \notin \cC
    \end{cases}$. Its conjugate is the \textit{support function} of $\cC$:
    \[
        \delta_{\cC}^*(x) = \sup_{z} \set{\ip{x, z} - \delta_{\cC}(z)}
        = \sup_{z \in \cC} \ip{x, z},
    \]
    where the simplification above occurs since $-\delta_{\cC}(z)$ is $-\infty$
    for $z \notin \cC$, which is a case that we disregard when computing the
    supremum, otherwise $0$.
\item For an indicator function $\delta_{\cC}$, its subgradient at $x$ is the
    so-called \textit{normal cone} at $x$, denoted by
    \[
        N_{\cC}(x) := \partial \delta_{\cC}(x).
    \]
    Its conjugate is defined as
    \[
        \partial \delta_{\cC}^*(y) := \argmax_{x \in \cC} \ip{x, y}.
    \]
\end{itemize}

\begin{exercise}{Conjugates}{indicator-conjugates}
    Prove the forms of the conjugates in the examples above.
\end{exercise}

The next proposition uncovers a relationship between the subgradients of the
original function and its Fenchel conjugate.
\begin{cproposition}{Subgradients of conjugates}{conj-subgrad}
    For a closed, convex, proper $f$, we have $y \in \partial f(x)$ exactly when
    $x \in \partial f^*(y)$. This is precisely when equality holds in the
    Fenchel-Young inequality.
\end{cproposition}
\begin{proof}
    The proof is left as an exercise.
\end{proof}

\begin{cdefinition}{Positive Homogeneity}{pos-homogen}
    We call a function $f: \Hbb \mapsto \overline{\Rbb}$ positively homogeneous
    if
    \[
        f(tx) = tf(x), \; \forall t \geq 0, \; \forall x,
    \]
    and, additionally, $f(0) = 0$.
\end{cdefinition}
The next proposition relates the positive homogeneity of support functions with
the set $\cC$ having certain desirable properties:
\begin{cproposition}{~}{support-fun-convex}
    The support function $f = \delta_{\cC}^*$ is proper, closed, convex and
    positively homogeneous when $\cC$ is a nonempty, closed, convex set that
    satisfies $\cC \neq \Hbb$. The converse is also true.
\end{cproposition}
\begin{proof}
    We prove each direction separately:
    \begin{itemize}
    \item $\Rightarrow$: the proof is immediate. \todo{Write down details}
    \item $\Leftarrow$: Let $\cC = \partial f(0)$. Given a $y \in \partial
    f(0)$, it holds that
    \[
        \ip{x, y} + \cancelto{0}{f(0)} \leq f(x), \; \forall x.
    \]
    Then, by the definition of the Fenchel conjugate:
    \[
        f^*(y) = \sup_{x} \set{\underbrace{\ip{x, y} - f(x)}_{\leq 0}}
        \leq 0,
    \]
    and equality is satisfied for $x = 0$. If $y \notin \partial f(0)$, then
    $\exists \bar{x}$ such that $\ip{\bar{x}, y} > f(\bar{x})$ (otherwise $y$
    would satisfy the subgradient inequality!). This implies
    \[
        f^*(y) \geq \sup_{t \geq 0} \set{\ip{t\bar{x}, y} - f(t \bar{x})}
        = \sup_{t \geq 0} t \set{\ip{\bar{x}, y} - f(\bar{x})} = +\infty
    \]
    This means that $f^* = \delta_{\cC}$ (\textbf{why?}) hence $f = f^{**} =
    \delta_{\cC}^*$.
    \end{itemize}
\end{proof}
The next theorem is the so-called \textit{max-formula} or what is also known as
steepest ascent.

\begin{ctheorem}{Max-formula}{max-formula}
	Take any proper convex function $f: \Hbb \mapsto \overline{\Rbb}$, and take
	$x \in \intr{\dom f}$. Consider its directional derivative
	\[
		f'(x; v) \triangleq \lim_{t \dto 0} \frac{f(x + tv) - f(x)}{t}.
	\]
	That limit exists and is equal to
	\[
		f'(x; v) = \max_{y \in \partial f(x)} \ip{y, v}.
	\]
	Furthermore, \[ \min_{v \in \mathbb{B}_2} f'(x; v) = -\mathrm{dist}(0,
	\partial f(x)) = - \norm{\partial^{\circ} f(x)} \] and is attained uniquely
	by $v = -\frac{\partial^{\circ} f(x)}{\norm{\partial^{\circ} f(x)}}$,
	assuming that $x$ is not a minimizer.
\end{ctheorem}
\begin{proof}
	Notice that in the above, we have that
	\( \sup_{y \in \partial f(x)} \ip{y, v} = \max_{y \in \partial f(x)} \ip{y,
	v}, \) since the subdifferential is nonempty at any point in the interior,
	and additionally closed, convex and bounded. Hence the supremum is
	attained, so we can write $\max$ instead of $\sup$.

	Now, let us proceed with the proof of the max-formula. By convexity, it is easy
	to check that $t \mapsto \frac{f(x + tv) - f(x)}{t}$ is nondecreasing in
	$t$, when $t \in (0, 1]$. Hence $f'(x; v) \geq 0$ and $0 \leq f'(x; v) -
	f(x)$.

	Now, it is again easy to check that $f'(x; \cdot)$ is finite by the
	Lipschitz property, convex since $f$ is convex, and continuous. By
	definition, it is positively homogeneous. (why?)
	Additionally, $\partial f'(x; 0) = \partial f(x)$, which is also easy to
	check. Then, the max formula follows by the previous result about support
	functions.

	For any $v \in \mathbb{B}_2$, we have
	\begin{align*}
		f'(x; v) &= \max_{y \in \partial f(x)} \ip{y, v} \geq \ip{g, v}
			\overset{(\text{Cauchy-Schwarz})}{\geq} - \norm{g},
	\end{align*}
	which is attained by $v = -\frac{g}{\norm{g}}$, where $g$ is the smallest
	norm vector in $\partial f(x)$. But, if $x$ is not a minimizer, we have
	that
	\begin{align*}
		f'\left(x; -\frac{g}{\norm{g}} \right) &= \max_{y \in \partial f(x)}
			\ip{y, -\frac{g}{\norm{g}}} \\
			&= -\frac{1}{\norm{g}} \min_{y \in \partial f(x)} \ip{y, g}
	\end{align*}
	Since $g$ is the shortest vector, by the characterization of projections we
	obtain that $\ip{g, y - g} \geq 0 \Rightarrow \ip{g, y} \geq \norm{g}^2$,
	so that $-\frac{1}{\norm{g}} \min_{y \in \partial f(x)} \ip{y, g} =
		- \norm{g}$.
\end{proof}


\section{Steepest Descent for Convex Functions}

Recall: for a proper convex function $f: \Hbb \to \overline{\Rbb}$, and a point
$x \in \intr{\dom f}$, the directional derivative
\[
	f'(x; v) = \lim_{t \dto 0} \frac{f(x + tv) - f(x)}{t}
\]
is uniquely minimized over $\mathbb{B}_2$ in the direction of the shortest
subgradient $v = -\frac{\partial^{\circ} f(x)}{\norm{\partial^{\circ} f(x)}}$,
assuming that $x$ is not a minimizer.

\textbf{Question}: is this the best we can do?

Recall the \textit{slope} of arbitrary $f: \Hbb \to \overline{\Rbb}$ at $x \in
\dom f$, defined by
\[
	\abs{\grad f}(x) = \begin{cases}
		0, & x \text{ local min} \\
		\limsup_{z \to x} \frac{f(x) - f(z)}{\norm{x - z}}, & \text{otherwise}
	\end{cases}
\]

\begin{exercise}{~}{slope-grad}
	Prove that if $f$ is differentiable we have $\abs{\grad f}(x) = \norm{\grad
	f(x)}$ at $x \in \dom f$.
\end{exercise}

The next theorem should be familiar from convex analysis:
\begin{ctheorem}{Sum rule}{sum-rule-subgrad}
	For convex, proper $f, g : \Hbb \mapsto \overline{\Rbb}$, if it holds that
	$\dom f \cap \intr{\dom g} \neq \emptyset$, then
	\[
		\partial (f + g) = \partial f + \partial g.
	\]
\end{ctheorem}

Using the sum rule for subgradients, we can prove the following Theorem, which
relates the slope with the shortest subgradient:

\begin{ctheorem}{~}{slope-shortest-subg}
	For proper convex $f: \Hbb \mapsto \overline{\Rbb}$ and $x \in \dom f$, we
	have that
	\begin{align*}
		\abs{\grad f}(x) &= \mathrm{dist}(0, \partial f(x)) \\
						 &= -\inf_{\norm{v}_2 = 1} f'(x; v),
	\end{align*}
	assuming that $x$ is not a minimizer.
\end{ctheorem}
\begin{proof}
	We will prove the above by showing two inequalities that can only be
	satisfied with equality.

	First, notice that for $v \in \mathbb{B}_2$:
	\[
		\abs{\grad f}(x) \geq \limsup_{t \dto 0} \frac{f(x) - f(x + tv)}{t}
		= - f^*(x; v)
	\]
	The first inequality above happens since letting $t \dto 0$ describes a
	subset of approaches $z \to x$ via the direction $v$. Therefore, we obtain
	that
	\[
		\abs{\grad f}(x) \geq -\inf_{\norm{v}_2 = 1} f'(x; v).
	\]

	Now, by the expansion of the subgradient identity, we have
	\begin{align*}
		\abs{\grad f}(x) &= \limsup_{z \to x} \frac{f(x) - f(z)}{\norm{x - z}}
					  \\ &\leq \limsup_{z \to x} \frac{\ip{y, x - z}}{\norm{x -
z}}, \; \forall y \in \partial f(x) \\
		&= \norm{y},
	\end{align*}
	since we are taking the $\limsup$, which gives us that $\abs{\grad f}(x)
	\leq \mathrm{dist}(0, \partial f(x))$ as the above holds for all elements
	of the subdifferential.

	Now it remains to prove that
	\begin{equation}
		\mathrm{dist}(0, \partial f(x)) \leq -\inf_{\norm{v}_2 = 1} f'(x; v).
		\label{eq:dist-dini-deriv}
	\end{equation}
	Choose any $\sigma \in (0, \mathrm{dist}(0, \partial f(x)))$. This implies
	that $0 \notin \partial f(x) + \sigma \mathbb{B}$ or, equivalently, that
	\[
		0 \notin \partial f(x) + \partial \left(\sigma \norm{\cdot -
		x}\right)(x),
	\]
	using properties of the Fenchel conjugate. By the sum rule, since $x \in
	\dom f$ and since the domain of $\norm{\cdot - x}$ is the whole space
	$\Hbb$, this implies that
	\[
		0 \notin \partial \left( f + \sigma \norm{\cdot - x} \right)(x),
	\]
	which shows that $x$ does not minimize that function. This implies that
	\[
		\exists z: f(z) + \sigma \norm{z - x} < f(x) \Rightarrow
		\frac{f(z) - f(x)}{\norm{z - x}} < -\sigma,
	\]
	which in turn implies (by the definition of the directional derivative)
	that $f'\left(x; \frac{z - x}{\norm{z - x}}\right) < -\sigma$. Since
	$\sigma$ was arbitrary, the claim~\eqref{eq:dist-dini-deriv} follows.

	Note that in the above, the sum rule does not assume $\emptyset \neq
	\partial f(x)$, so $\partial f(x)$ can be empty. Even in that case, the proof
	still goes through.
\end{proof}
Theorem~\ref{thm:slope-shortest-subg} implies that we can achieve instantaneous
decrease arbitrarily close to the slope by looking along rays, for convex
functions. Rays are only one of the many ways to approach $x$ in the definition
of the slope.

The following proposition generalizes the well-known chain rule from calculus:
\begin{cproposition}{Chain rule}{slope-chain-rule}
	For arbitrary $f$ and a $C^1$ function $\varphi: (a, b) \to \Rbb$ with
	$\varphi' > 0$, we know that if $a < f(x) < b$, it holds that
	\[
		\abs{\grad (\varphi \circ f)}(x) = \varphi'(f(x)) \abs{\grad f}(x).
	\]
\end{cproposition}
\begin{proof}
	We will show the above by proving two inequalities in opposite directions:
	\begin{enumerate}
		\item[$(\leq):$] wlog, assume that $\abs{\grad (\varphi \circ f)}(x) > 0$
			and $\varphi'(f(x)) \abs{\grad f}(x) < \infty$. Choose a sequence
			$\set{x_n}_{n=1}^{\infty}$, $x_n \to x$ that satisfies
			\[
				\frac{\varphi(f(x)) - \varphi(f(x_n))}{\norm{x - x_n}}
				\to \abs{\grad (\varphi \circ f)}(x).
			\]
			Since $\varphi$ is increasing, eventually we will have
			\[
				\varphi(f(x)) > \varphi(f(x_n)) \Rightarrow
				f(x) > f(x_n),
			\]
			otherwise the LHS would be zero or negative, contrary to our
			assumption. Furthermore, $f(x_n) \to f(x)$, since otherwise
			$\abs{\grad f}(x) \to \infty$. By the Mean Value Theorem
			\begin{align*}
				\exists w_n \in (f(x_n), f(x)): & \frac{\varphi(f(x)) -
				\varphi(f(x_n))}{\norm{x - x_n}} =
				\frac{\varphi(f(x)) - \varphi(f(x_n))}{f(x) - f(x_n)}
				\cdot \frac{f(x) - f(x_n)}{\norm{x - x_n}} \\
				&= \varphi'(w_n) \frac{f(x) - f(x_n)}{\norm{x - x_n}}.
			\end{align*}
			But, $w_n \to f(x)$ and $\frac{f(x) - f(x_n)}{\norm{x - x_n}} \leq
			\abs{\grad f}(x)$.
		\item[$(\geq):$] by the arguments above, we can write
			\[
				\abs{\grad f}(x) = \abs{\grad(\varphi^{-1} \circ \varphi \circ
				f)}(x), \]
			where the inverse $\varphi^{-1}$ exists by the differentiability
			and strict monotonicity of $\varphi$. This gives us
			\begin{align*}
				\abs{\grad f(x)} &\leq \left( \varphi^{-1} \right)'(\varphi(f(x)))
					\abs{\grad (\varphi \circ f)}(x) \\
					&= \frac{1}{\varphi'(f(x))} \abs{\grad (\varphi \circ
				f)}(x),
			\end{align*}
			so the claim follows. In the last expression, we used the fact that
			\(
				(g^{-1})'(g(x)) = \frac{1}{g'(x)},
			\) for any $1-1$, differentiable function (here $\varphi$ since
			diffable and strictly monotone), by the Inverse Function Theorem.
	\end{enumerate}
\end{proof}
The final question we are asking is the following: given a closed, proper,
convex $f: \Hbb \to \overline{\Rbb}$, and an initial point $x_0 \in \Hbb$, is
there a trajectory starting at $x_0$ that always attains the best possible
instantaneous rate of decrease? In 1973, Brezis answered this question.

\begin{ctheorem}{~}{brezis-rate}
	There exists an absolutely continuous trajectory $X(t)$ starting at $x_0$,
	meaning
	\[
		X(t) = x_0 + \int_0^t W(s) \dd s, \; W \text{ integrable},
	\]
	such that:
	\begin{itemize}
		\item $X(0) = x_0$, and
		\item $ \frac{\dd X}{\dd t} \in -\partial f(X(t)) $ almost everywhere.
	\end{itemize}
	This trajectory attains the maximum rate of decrease.
\end{ctheorem}


\section{Steepest Descent Trajectories}

Let us recall a definition that we touched upon in previous lectures. Given
proper $f: \Hbb \to \overline{\Rbb}$, $x_0 \in \dom f$, a steepest descent
trajectory is an \textit{absolutely continuous} curve $X(\cdot): \Rbb_+ \to
\Hbb$, $X(0) = x_0$, such that
\begin{equation}
	X(t) = x_0 + \int_0^t W(s) \dd s, \;
	\frac{ \dd{~}^+ }{\dd{t}} f(X(t)) = -
	\underbrace{\abs{\grad f}(X(t))}_{\text{slope}} \cdot
	\underbrace{\norm{ \frac{\dd{~}^+}{\dd{t}} X(t) }}_{\text{speed}}
	\label{eq:steepest-descent-trajectory}
\end{equation}
In plaintext, the above simply says that the rate by which the function value
decreases along a steepest descent trajectory is proportional to the slope
times speed.

\begin{cproposition}{}{}
	If $X(\cdot)$ is a steepest descent trajectory for $f$, then it is also a
	steepest descent trajectory for $\phi \circ f$, where $\phi \in C^1$ and
	$\phi' > 0$.
\end{cproposition}
\begin{proof}
	Using the ordinary chain rule, we write
	\begin{align*}
		\frac{\dd{~}^+}{\dd{t}} \left(\phi \circ f\right)(X(t)) &=
			\phi'(f(X(t))) \frac{\dd{~}^{+}}{\dd{t}} f(X(t)) \\
			&= -\phi'(f(X(t))) \cdot \abs{\grad f}(X(t))
			   \norm{\frac{\dd{~}^{+}}{\dd{t}} X(t)} \\
			&\overset{(\text{Prop.}~\ref{prop:slope-chain-rule})}{=}
			-\grad(\phi \circ f)(X(t)) \norm{\frac{\dd{~}^{+}}{\dd{t}} X(t)},
	\end{align*}
	which concludes that
	\[
		\frac{\dd{~}^+}{\dd{t}} (\phi \circ f)(X(t))
		= -\grad (\phi \circ f)(X(t)) \norm{\frac{\dd{~}^{+}}{\dd{t}} X(t)}.
	\]
\end{proof}

Now, let us suppose that $f$ is \textbf{sharp} along $X(t)$, i.e. $\exists
\varepsilon > 0$ such that $\abs{\grad f}(X(\cdot)) \geq \varepsilon$. Then,
for any $0 \leq t < s$, the length of the curve $\ell(X(\cdot), t, s)$ satisfies
\begin{align*}
	\ell(X, t, s) &= \int_t^s \norm{\dot{x}}
	\overset{\text{\cref{eq:steepest-descent-trajectory}}}{=}
	\int_t^s -\frac{\frac{\dd{~}^{+}}{\dd{\tau}} f(X(\tau))}{\abs{\grad
	f}(X(t))}
		\dd{\tau} \\
		&\leq \frac{1}{\varepsilon} \int_t^s
			-\frac{\dd{~}^+}{\dd{\tau}} f(X(\tau)) \dd{\tau} \\
		&=\frac{1}{\varepsilon} \left(f(X(t)) - f(X(s))\right),
\end{align*}
where in the above we've made use of the fact that $\frac{\dd{~}^{+}}{\dd{t}}
f(X(t))
= \frac{\dd{~}}{\dd{t}} f(X(t))$ \textit{almost everywhere} to perform the
integration, by replacing the integral with the one involving the ordinary
derivative.
Therefore, sharpness gives a uniform bound on the length of steepest descent
trajectories in terms of decrease in function value.

Furthermore, if we know that $f(X(t)) \overset{t \to \infty}{\to} \min f$, and
assuming the minimum is attained, the uniform bound above is enough to prove
that the trajectory (or sequence, if discretized) above has the Cauchy
property, hence converges in norm. Recall that a sequence
$\set{x_n}_{n=1}^{\infty}$ is Cauchy if $\forall \varepsilon > 0$, $\exists N$
such that $\abs{x_m - x_n} < \varepsilon, \; \forall m, n > N$.


\section{KL Property and Algorithms}

Consider $f : \Hbb \to \overline{\Rbb}$ proper, with $\min f$ finite.

\begin{cdefinition}{Kurdyka-{\L}ojasiewicz Property}{kl-prop}
	The Kurdyka-Lojasiewicz property holds for $f$ on $\cX \subseteq \Hbb$
	means that $\exists \rho > 0$ and a continuous, concave function
	\[
		\phi: [0, \rho) \mapsto \Rbb_+, \; \phi(0) = 0, \; \phi' > 0,
	\]
	called the \textbf{desingularization} function, such that
	\begin{align*}
		\abs{\grad (\phi \circ (f - \min f))}(x) &\geq 1, \quad
		\forall x : 0 < f(x) - \min f < \rho
	\end{align*}
	Typically, $\phi(s) = k s^{1 - \theta}, \; \theta \in [0, 1)$, and $\theta$
	is sometimes called the \textit{{\L}ojasiewicz exponent}.
\end{cdefinition}
Notice that the KL property implies sharpness for strictly convex quadratics.

In general, we will consider \textit{slope descent} sequences $\set{x_k} \in
\Hbb$ that satisfy the following properties:
\begin{enumerate}
	\item (sufficient decrease): $\exists \alpha > 0$ such that
		\[
			f(x_k) - f(x_{k+1}) \geq \alpha \norm{x_k - x_{k+1}}^2.
		\]
		It should be possible to get such a decrease if we take sufficiently
		small steps, because it's relatively easy to guarantee by construction
		of our optimization algorithms.
	\item (error bound): $\exists \beta > 0$ such that
		\[
			\abs{\grad f}(x_{k+1}) \leq \beta \norm{x_k - x_{k+1}}.
		\]
\end{enumerate}
Notice that, for sharp functions, the second property would imply that any
algorithm that generates such a descent sequence must terminate. This is
because the slope is bounded from below for sharp functions, which means that
the size of the steps of the algorithm cannot be arbitrarily small. The only
way this can happen is if the algorithm terminates.

\paragraph{Example:} Consider closed convex proper $f: \Hbb \to
\overline{\Rbb}$. $\gph f = \set{(x, y) \mmid y \in \partial f(x)}$ is
closed (norm-weak). This means that
\[
	\left. \begin{array}{c}
		x_k \overset{\text{(norm)}}{\to} x \\
		y_k \overset{\text{(weak)}}{\to} y \\
		y_k \in \partial f(x_k)
	\end{array} \right\} \Rightarrow y \in \partial f(x),
\]
and $f(x_k) \to f(x)$. This is what is called \textit{subdifferential
continuity}.

\begin{ctheorem}{Convergence under KL}{KL-convergence-convex}
	Suppose $\set{x_k}$ is a slope descent sequence for a closed convex proper
	$f: \Hbb \to \overline{\Rbb}$ with $\min f$ finite, and assume that the KL
	property holds on $\set{x_k}$. Then
	\begin{align*}
		x_k & \overset{\text{(norm)}}{\to} x^* \\
		f(x_k) & \dto \min f \\
		\norm{x_k - x^*} &\leq \frac{\beta}{\alpha} \phi(f(x_k) - \min f)
			+ \sqrt{\frac{f(x_{k-1}) - \min f}{\alpha}},
	\end{align*}
	where $x^*$ is a minimizer of $f$.
\end{ctheorem}
\begin{proof}
	Assume $\min f = 0$, so that $f(x) - \min f = f(x), \; \forall x$.
	Let us consider the difference $\phi(f(x_k)) - \phi(f(x_{k+1}))$. By the
	concavity of $\phi$ (and its corresponding subgradient inequality), we have
	\begin{align*}
		\phi(f(x_k)) - \phi(f(x_{k+1})) &\geq \phi'(f(x_k)) \left(
			f(x_k) - f(x_{k+1}) \right) \geq \alpha \phi'(f(x_k))
				\norm{x_k - x_{k+1}}^2 \\
			&\geq \alpha \frac{\norm{x_k - x_{k+1}}^2}{\abs{ f }(x_k)}
			 \quad \text{(by KL Prop. and chain rule)} \\
			&\geq \frac{\alpha}{\beta} \frac{\norm{x_k -
			x_{k+1}}^2}{\norm{x_{k-1}
			- x_{k}}} \\
			&\geq \frac{\alpha}{\beta} \frac{2\norm{x_k - x_{k+1}}
			 \norm{x_{k-1} - x_k} - \norm{x_{k-1} - x_k}^2}{
				 \norm{x_{k-1} - x_k}},
	\end{align*}
	where the last line follows since
	\( \left( \norm{x_k - x_{k+1}} - \norm{x_{k-1} - x_k} \right)^2 \geq 0 \),
	so we obtain
	\begin{align*}
		\phi(f(x_k)) - \phi(f(x_{k+1})) &\geq
			\frac{\alpha}{\beta} \left(2 \norm{x_k - x_{k+1}} - \norm{x_{k-1} -
			x_k} \right).
	\end{align*}
	Define $\lambda_k := \phi(f(x_k)) + \frac{\alpha}{\beta} \norm{x_{k-1} -
	x_k}$. Thus we have proved that
	\[
		\lambda_k - \lambda_{k+1} \geq \frac{\alpha}{\beta} \norm{x_k -
		x_{k+1}}, \; \forall k.
	\]
	We now resort to a telescoping sum argument. Summing over all $k$'s, we
	conclude that $\sum_{k=0}^{\infty} \norm{x_k - x_{k+1}}$ is convergent,
	hence $\set{x_k}$ is a Cauchy sequence, converging strongly to some $x^*$.

	Furthermore, by the property (2) and the fact that $\abs{\grad f}(x) =
	\mathrm{dist}(0, \partial f(x))$, we know that $\partial^{\circ} f(x_k) \to 0$,
	and since the subdifferential is closed, $0 \in \partial f(x^*)$. Hence $x^*$
	minimizes $f$ and $f(x_k) \dto \min f$.

	More generally, if we look at the distance between iterates, $\norm{x_k -
	x_{m}}$, we have (by triangle inequality and telescoping sums):
	\begin{align*}
		\norm{x_k - x_{m+1}} &\leq \sum_{j=k}^m \norm{x_j - x_{j+1}} \\
			&\leq \frac{\beta}{\alpha} \sum_{k}^m \left(
				\lambda_j - \lambda_{j+1} \right) \overset{m \to
				\infty}{\Rightarrow} \\
		\norm{x_k - x^*} &\leq \frac{\beta}{\alpha} \left(
			\phi(f(x_k)) + \frac{\alpha}{\beta}
			\norm{x_{k-1} - x_k}
		\right) \\
		&= \frac{\beta}{\alpha} \phi(f(x_k)) +
		\sqrt{\frac{f(x_{k-1}) - f(x_k)}{\alpha}}
		\leq \frac{\beta}{\alpha} \phi(f(x_k)) +
		\sqrt{\frac{f(x_{k-1})}{\alpha}},
	\end{align*}
	where we used the fact that $\min f = 0 \Rightarrow
	f(x_{k-1}) - f(x_k) \leq f(x_{k-1})$. We've used the convexity of $f$
	indirectly above: for example, the fact that $\abs{\grad f} = \mathrm{dist}(0,
	\partial f(x))$, or the convergence to a minimizer implying convergence to
	the global minimum of $f$. However, convexity is not absolutely necessary
	for these conditions.
\end{proof}

For the sequel, we will need the following proposition:
\begin{cproposition}{Quadratic upper bound}{quad-upper-bound-kl}
	Suppose that $f: \Hbb \to \Rbb$ is convex with $L$-Lipschitz $\grad f$.
	Then we have that
	\[
		f(z) \leq f(x) + \ip{\grad f(x), z - x} + \frac{L}{2} \norm{z - x}^2,
		\; \forall z.
	\]
\end{cproposition}

Let us now talk about smooth convex minimization. Many algorithms for
minimizing functions as in~\cref{prop:quad-upper-bound-kl} proceed iteratively
at the current $x$ by choosing $x^+$ to satisfy
\[
	x^+ \in \argmin_z f(x) + \ip{\grad f(x), z - x} +
		\frac{1}{2} \ip{A (z - x), z - x} =
		\argmin_z \ip{\grad f(x), z - x} +
		\frac{1}{2} \ip{A (z - x), z - x},
\]
where $A^\top = A, \; A$ psd, giving us
\[
	x^+ = x - A^{-1} \grad f(x).
\]
Algorithms like Newton, quasi-Newton, steepest descent, trust regions methods
are subsumed by this framework. The following exercise is easy to check:
\begin{exercise}{}{}
	Providing that $\set{A_k}_{k=1}^{\infty}$ used in the above optimization
	steps is uniformly bounded and $\norm{A^{-1}} \leq \frac{C}{L}$, $C$ a
	constant (typically $2$), then the sequence of iterates $\set{x_k}$
	satisfies the postulates of a slope descent sequence.
\end{exercise}


\section{KL and complexity}

Recall that we are mostly concerned with $f: \Hbb \to \overline{\Rbb}$, closed,
proper and convex, and assume that $\min f = 0$ for simplicity.

\textbf{Slope descent sequence}: a sequence $\set{x_k}_{k=1}^{\infty}$
satisfying the following properties $\forall k \in \Nbb$:
\begin{enumerate}
	\item $f(x_k) - f(x_{k+1}) \geq \alpha \norm{x_k - x_{k+1}}^2$, for a
	universal constant $\alpha > 0$
	\item $\abs{\grad f}(x_{k+1}) \leq \beta \norm{x_k - x_{k+1}}$, for a
	universal constant $\beta > 0$.
\end{enumerate}

Assume that the KL property is satisfied: $\exists$ desingularizing function
$\phi: [0, \rho) \to \Rbb_+$, continuous and concave, with $\phi(0) = 0$ and
$\phi' > 0, $ continuous on $(0, \rho)$, (e.g. $\phi(s) = ks^{1- \theta},
0 \leq \theta < 1$), such that
\[
	\abs{\grad (\phi \circ f)}(x) \geq 1, \; \forall x \in \cX,
\]
where $\cX$ is the slope descent sequence.
Then we deduce that $x_k \overset{\mathrm{norm}}{\to} x^*$, where $x^*$ is a
minimizer of $f$, and
\[
	\norm{x_k - x^*} \leq \frac{\beta}{\alpha} \phi(f(x_k))
		+ \sqrt{\frac{f(x_{k-1})}{\alpha}}.
\]

By inverse function Theorem, we know that $\phi$ has an inverse $\psi :=
\phi^{-1}$, defined on some interval $(0, \mu) \to \Rbb_+$, for some $0 < \mu <
\phi(\rho)$, which will be:
\begin{itemize}
	\item continuous, with $\psi(0) = 0$, and
	\item $\psi' > 0$, continuous on $(0, \mu)$, with
		\[
			\psi'(\phi(s)) = \frac{1}{\phi'(x)}.
		\]
\end{itemize}
Since $\phi$ is concave, it follows that $\psi$ is convex.
\textbf{Notice}: $\phi$ desingularizes $\psi$. So it makes sense, in an
exploratory sense, to try and apply our theorem to $\psi$.

We need an \textit{additional} assumption: $\psi'(0) = 0$ and $\psi'$ is also
L-Lipschitz on $[0, \mu)$.

\paragraph{Example:} for $\phi(s) = ks^{1 - \theta}$, we obtain $\psi(t) = k'
t^{\frac{1}{1 - \theta}}, t \geq 0$. A canonical example is when $\theta =
\frac{1}{2}$, where we obtain $\psi(t) = k' t^2, \; t \geq 0$.

We will see that the complexity depends just on $\alpha, \beta, L$ and $\psi$,
by applying the proximal point method to minimize $\psi$.

\subsubsection{Proximal point method}
Recall: for closed, proper, convex $f: \Hbb \to \overline{\Rbb}$ and any $x \in
\dom f$, $\exists! y \in \Hbb$ such that
\[
	y = \argmin_x \set{f(x) + \frac{1}{2} \norm{x - y}^2},
\]
characterized by (using the sum rule for the subdifferential):
\[
	x - y \in \partial f(y),
\]
which has unique solution $y^*$, and this solution is called the
\textit{proximal map}. The proximal map is $1$-Lipschitz.

E.g. $f = \delta_C$, $C$ closed and convex, then $\prox{f}{~} = \cP_{C}$, where
$\cP$ denotes the projection operator.

If $\argmin f \neq \emptyset$, then the proximal point method
\[
	x_{k+1} = \prox{\lambda f}{x_k}, \; \lambda > 0,
\]
converges to a minimizer (weakly) and $f(x_k) \dto \min f$ at rate
$\cO\left(\frac{1}{k}\right)$. In fact, we can choose $\lambda = \lambda_k$ at
every step, providing $\sum_k \lambda_k = \infty$.


Now, let us apply the proximal point method to $\psi$. Choose $t_k > 0$, so
that $\psi(t_k) = f(x_k)$, implying that $t_k = \psi^{-1}(f(x_k)) =
\phi(f(x_k))$, and choose $\lambda_k > 0$ so that
\[
	t_{k+1} = \prox{\lambda_k \psi}{t_k}, \; \forall k.
\]
By definition, this means that
\[
	\lambda_k \psi'(t_{k+1}) + t_{k+1} - t_k = 0 \Rightarrow
	\lambda_k = \frac{t_k - t_{k+1}}{\psi'(t_{k+1})}.
\]
Let us try and get a handle on how this sequence of $\lambda_k$ behaves. Since
$\set{x_k}$ is a slope descent sequence. If we look at
$\frac{\alpha}{\beta^2}$, we obtain
\[
	\frac{\alpha}{\beta^2} \leq \frac{f(x_k) - f(x_{k+1})}{\abs{\grad
	f}^2(x_{k+1})}.
\]
The KL inequality says that $1 \leq \abs{\grad (\phi \circ f)}(x_{k+1}), \; \forall k$, which by the chain rule gives us
\begin{align*}
	1 & \leq \phi'(f(x_{k+1})) \cdot \abs{\grad f}(x_{k+1}) \Rightarrow
	\abs{\grad f}(x_{k+1}) \geq \frac{1}{\phi'(f(x_{k+1}))} \\
	&= \frac{1}{\phi'(\psi(t_{k+1}))} = \psi'(t_{k+1})
\end{align*}
Hence we obtain that
\begin{align*}
	\frac{\alpha}{\beta^2} &\leq \frac{\psi(t_k) - \psi(t_{k+1})}
		{\left(\psi'(t_{k+1}) \right)^2} \\
		&\overset{\cref{prop:quad-upper-bound-kl}}{\leq} \frac{\psi'(t_{k+1})(t_k - t_{k+1}) - \frac{L}{2}\norm{t_k -
		t_{k+1}}^2}{\psi'(t_{k+1})} \\
		&= \lambda_k + \frac{L}{2} \lambda_k^2 \Rightarrow \\
	\lambda_k^2 + \frac{2}{L} \lambda_k + \frac{1}{L^2} &\geq
	\frac{2\alpha}{L\beta^2} + \frac{1}{L^2} \Rightarrow
	\lambda_k \geq
	\underbrace{\sqrt{\frac{2a}{L\beta^2} + \frac{1}{L^2}} - \frac{1}{L}}_{=:
	\bar{\lambda} > 0}.
\end{align*}
The last step involves proving the following, which is an easy exercise:
\begin{exercise}{}{}
	Define $\tau_0 = t_0$ and
	\begin{align*}
		\tau_{k+1} &= \prox{\bar{\lambda} \psi}{\tau_k} \\
		t_{k+1} &= \prox{\lambda_k \psi}{t_k}, \; \lambda_k \geq
		\bar{\lambda},
	\end{align*}
	for any convex $\psi$ with $\psi'(0) = 0$. Then $t_k \leq \tau_k, \;
	\forall k$.
	\begin{proof}
		An easy inductive argument.
	\end{proof}
\end{exercise}

Define $\tau_0 = f(x_0)$ and $\tau_{k+1} = \prox{\bar{\lambda} \psi}{\tau_k},
\; \forall k$. Then
\[
	\norm{x_k - x^*} \leq \frac{\beta}{\alpha} \tau_k +
	\sqrt{\frac{\psi(\tau_{k-1})}{\alpha}}.
\]
In particular, if $\phi(s) = k \sqrt{s}$, which implies that $\psi(s) = k'
s^2$, the proximal point method gives \textit{linear convergence} of $x_k$ to
$x^*$.

In order to apply the above to a real problem we need:
\begin{itemize}
	\item an algorithm, which we also need to be able to prove that satisfies
	the slope descent sequence postulates (get a hold of $\alpha, \beta$).
	\item a guarantee about the KL property and the existence of a
	desingularizing function. We'll take about finding such functions in the
	sequel.
	\item look at the behavior of $\tau_k$ to see how fast it is converging to
	zero, assuming the sequence $f(x_k)$ is precomputed.
\end{itemize}


\section{Slope Descent Sequences}
Recall $f: \Hbb \to \overline{\Rbb}$ a closed, convex, proper function
satisfying (for $\alpha, \beta > 0$):
\begin{align}
	\left\{ \begin{aligned}
	f(x) - f(x_+) &\geq \alpha \norm{x - x_+}^2 \\
	\abs{\grad f}(x_+) &\leq \beta \norm{x - x_+}
	\end{aligned} \right.
	\label{eq:slope-descent-recap}
\end{align}

In this lecture, we are going to talk about ensuring the slope descent sequence
properties hold and also about ensuring that the Kurdyka-{\L}ojasiewicz
condition holds, through the scope of the proximal gradient algorithm.

\subsection{Proximal gradient method}
Recall that this method is also known as forward-backward splitting. Suitable
for problems of the form
\begin{align}
	\min_{x \in \Hbb} \set{g(x) + h(x)}
	\label{eq:prox-grad-prob}
\end{align}
where $g$ is closed, convex and nonsmooth and $h$ is convex with $\grad h$
being $L$-Lipschitz.

At current $x$, appealing to the quadratic upper bound implies that
\[
	g(z) + h(z) \leq g(z) + h(x) + \ip{\grad h(x), z - x}
	+ \frac{1}{2 \lambda} \norm{z - x}^2,
\]
for a small enough $\lambda$. Note that we keep $g(z)$ instead of $g(x)$ on the
RHS since we are only majorizing the smooth part.

Now, choose $x_+$ to minimize the RHS, or equivalently
\[
	\lambda g(z) + \frac{1}{2} \norm{z - (x - \lambda \grad h(x))}^2,
\]
which implies that $x_+ = \overbrace{\prox{\lambda g}{
\underbrace{x - \lambda \grad h(x)}_{\text{forward step}}}}^{\text{backward
step}}$; hence the name forward-backward splitting.

By definition, this gives us
\[
	0 \in \partial g(x_+) - \grad h(x) + \frac{1}{\lambda} (x_+ - x).
\]
Using the above, notice that
\begin{align*}
	f(x) - f(x_+) &= g(x) - g(x_+) + h(x) - h(x_+) \geq
	\ip{\frac{1}{\lambda} (x_+ - x) - \grad h(x), x - x_+}
	+ \ip{\grad h(x), x - x_+} - \frac{L}{2} \norm{x - x_+}^2 \\
	&= \left( \frac{1}{\lambda} - \frac{L}{2} \right) \norm{x - x_+}^2.
\end{align*}
The condition above is precisely the sufficient decrease condition
in~\cref{eq:slope-descent-recap}, providing we always choose $\lambda$ such
that $\lambda \leq \bar{\lambda} < \frac{2}{L}$.

Additionally,
\begin{align*}
	\frac{1}{\lambda} (x - x_+) - \grad h(x) &\in \partial g(x_+) \Rightarrow
	\frac{1}{\lambda} (x - x_+) + \grad h(x_+) - \grad h(x) \in \grad h(x_+) +
	\partial g(x_+) \\
	\Rightarrow \abs{\grad f}(x_+) &\leq 
	\norm{\frac{1}{\lambda} (x - x_+) + \grad h(x_+) - \grad h(x)} \\
	&\leq \left( \frac{1}{\lambda} + L \right) \norm{x - x_+},
\end{align*}
where in the last inequality we've used the triangle inequality and the fact
that $\grad h$ is $L$-Lipschitz. Additionally, we've appealed to the fact that
$\abs{\grad f}$ is the length of the shortest subgradient, so the subgradient
we've identified above is an upper bound for it.
Therefore, providing we choose $\lambda \geq
\underline{\lambda} > 0$, we satisfy the second of the postulates
in~\cref{eq:slope-descent-recap} with $\beta = \frac{1}{\underline{\lambda}} +
L$.

The above is an example of verifying slope descent sequence properties for an
optimization algorithm.


\begin{ctheorem}{Moreau-Rockafellar}{moreau-rockafellar}
	Consider $f: \Hbb \to \overline{\Rbb}$ a closed, proper, convex function.
	Then, $\forall x \in \Hbb$,
	\[
		e_f(x) \triangleq \inf_{y \in \Hbb} \set{f(y) + \frac{1}{2} \norm{y - x}^2}
	\]
	is attained by a unique $y = \prox{f}{x}$ and characterized by
	\[
		x - y \in \partial f(y).
	\]
	The quantity $e_f(x)$ is called the \textit{Moreau envelope} of $f$ at $x$.
	Additionally, $e_f(x)$ is convex, everywhere finite, and smooth, with
	$\grad e_f (x) = I - \prox{f}{\cdot}$. Moreover, $\grad e_f, \prox{f}{\cdot}, 2
	\prox{f}{\cdot} - I$ are all nonexpansive (1-Lipschitz).
\end{ctheorem}

\textbf{Aside}: $\frac{1}{\lambda} e_{\lambda f}(\cdot) \uparrow f(\cdot)$ as
$\lambda \dto 0$.

Let us recall a previous theme: Brezis' argument for solvability of $\dot{x}(t)
\in -\partial f(x(t))$, which is the subgradient descent trajectory, works by
approximating
\[
	\dot{x}(t) = -\grad \left( \frac{1}{\lambda} e_{\lambda f} \right) x(t),
	\; t \geq 0,
\]
which is reduced to solving ODEs that involve smooth, Lipschitz functions, for
which well-established arguments apply.


\subsection{Examples}

\paragraph{1. Closest pair of convex sets.}
Consider sets $C, D \subseteq \Hbb, \; C, D \neq \emptyset$, both closed and
convex. If $C \cap D \neq \emptyset$, we are seeking a point in the
intersection.

Equivalently:
\[
	\inf_{x} \set{\delta_C(x) + \frac{1}{2} \mathrm{dist}^2_D(x)}, \;
	\mathrm{dist}_D(x) = \inf_{y \in D} \norm{x - y}.
\]
We claim that the above is amenable to the proximal gradient method, since
$\delta_C(x)$ is convex and nonsmooth and $\mathrm{dist}^2_D(x)$ is the Moreau
envelope of $\delta_D(x)$, verifiable by bringing it to the form
\[
	\frac{1}{2} \mathrm{dist}^2_D(x) = \inf_{y} \set{\delta_D(y) + \frac{1}{2}
	\norm{x - y}^2}.
\]
Hence we obtain
\begin{align*}
	\grad h(x) &= \grad e_{\delta_D}(x) = x - \prox{\delta_D}{x} \\
		&= x - \mathrm{proj}_{D}(x).
\end{align*}
Additionally, $\grad h(x)$ has Lipschitz constant $1$ by virtue of being the
gradient of a Moreau envelope~\cref{thm:moreau-rockafellar}. Therefore, the
iteration becomes
\begin{align}
	x_+ &\gets \mathrm{prox}_{\lambda g}\left( x - \lambda \grad h(x) \right) \\
		&= \cP_C \left( x - \lambda (x - \cP_D(x)) \right) \\
		&= \cP_C \bigg( \underbrace{(1 - \lambda) x + \lambda
		\cP_D(x)}_{\text{relaxed projection}} \bigg)
\end{align}
In particular, for $\lambda = 1$, we recover the traditional method of
alternating projections. In terms of slope descent sequence properties,
we recover
\[
	\alpha = \frac{1}{\lambda} - \frac{1}{2}, \; \beta = \frac{1}{\lambda} + 1.
\]

\paragraph{2. $\ell_1$-regularized least squares}
In this case, our problem is
\[
	\inf_{x \in \Rbb^n} \set{\mu \norm{x}_1 + \frac{1}{2} \norm{Ax - b}^2_2},
	\; \mu > 0, \; A \in \Rbb^{m \times n}, \; b \in \Rbb^m.
\]
Adapting to our forward-backward framework, we have $g(x) = \mu \norm{x}_1$,
and $h(x) = \frac{1}{2} \norm{Ax - b}_2^2$, with
\[
	\grad h(x) = A^\top (Ax - b) \Rightarrow
	\abs{h(x) - h(y)} \leq \opnorm{A^\top A}.
\]
The proximal gradient method for this problem reads
\begin{align*}
	x_+ &\gets \prox{\mu \norm{\cdot}_1}{x - \lambda A^\top (A x - b)} \\
		&= \sign\left(x - \lambda A^\top (A x - b)\right) \cdot
			\left( \abs{x - A^\top (A x - b)} - \mu \right)_+,
\end{align*}
which recovers the elementwise soft thresholding operator on a steepest descent
step.


\appendix

\clearpage
\bibliographystyle{plain}
\bibliography{references}
\end{document}
