\section{Steepest Descent Trajectories}

Let us recall a definition that we touched upon in previous lectures. Given
proper $f: \Hbb \to \overline{\Rbb}$, $x_0 \in \dom f$, a steepest descent
trajectory is an \textit{absolutely continuous} curve $X(\cdot): \Rbb_+ \to
\Hbb$, $X(0) = x_0$, such that
\begin{equation}
	X(t) = x_0 + \int_0^t W(s) \dd s, \;
	\frac{ \dd{~}^+ }{\dd{t}} f(X(t)) = -
	\underbrace{\abs{\grad f}(X(t))}_{\text{slope}} \cdot
	\underbrace{\norm{ \frac{\dd{~}^+}{\dd{t}} X(t) }}_{\text{speed}}
	\label{eq:steepest-descent-trajectory}
\end{equation}
In plaintext, the above simply says that the rate by which the function value
decreases along a steepest descent trajectory is proportional to the slope
times speed.

\begin{cproposition}{}{}
	If $X(\cdot)$ is a steepest descent trajectory for $f$, then it is also a
	steepest descent trajectory for $\phi \circ f$, where $\phi \in C^1$ and
	$\phi' > 0$.
\end{cproposition}
\begin{proof}
	Using the ordinary chain rule, we write
	\begin{align*}
		\frac{\dd{~}^+}{\dd{t}} \left(\phi \circ f\right)(X(t)) &=
			\phi'(f(X(t))) \frac{\dd{~}^{+}}{\dd{t}} f(X(t)) \\
			&= -\phi'(f(X(t))) \cdot \abs{\grad f}(X(t))
			   \norm{\frac{\dd{~}^{+}}{\dd{t}} X(t)} \\
			&\overset{(\text{Prop.}~\ref{prop:slope-chain-rule})}{=}
			-\grad(\phi \circ f)(X(t)) \norm{\frac{\dd{~}^{+}}{\dd{t}} X(t)},
	\end{align*}
	which concludes that
	\[
		\frac{\dd{~}^+}{\dd{t}} (\phi \circ f)(X(t))
		= -\grad (\phi \circ f)(X(t)) \norm{\frac{\dd{~}^{+}}{\dd{t}} X(t)}.
	\]
\end{proof}

Now, let us suppose that $f$ is \textbf{sharp} along $X(t)$, i.e. $\exists
\varepsilon > 0$ such that $\abs{\grad f}(X(\cdot)) \geq \varepsilon$. Then,
for any $0 \leq t < s$, the length of the curve $\ell(X(\cdot), t, s)$ satisfies
\begin{align*}
	\ell(X, t, s) &= \int_t^s \norm{\dot{x}}
	\overset{\text{\cref{eq:steepest-descent-trajectory}}}{=}
	\int_t^s -\frac{\frac{\dd{~}^{+}}{\dd{\tau}} f(X(\tau))}{\abs{\grad
	f}(X(t))}
		\dd{\tau} \\
		&\leq \frac{1}{\varepsilon} \int_t^s
			-\frac{\dd{~}^+}{\dd{\tau}} f(X(\tau)) \dd{\tau} \\
		&=\frac{1}{\varepsilon} \left(f(X(t)) - f(X(s))\right),
\end{align*}
where in the above we've made use of the fact that $\frac{\dd{~}^{+}}{\dd{t}}
f(X(t))
= \frac{\dd{~}}{\dd{t}} f(X(t))$ \textit{almost everywhere} to perform the
integration, by replacing the integral with the one involving the ordinary
derivative.
Therefore, sharpness gives a uniform bound on the length of steepest descent
trajectories in terms of decrease in function value.

Furthermore, if we know that $f(X(t)) \overset{t \to \infty}{\to} \min f$, and
assuming the minimum is attained, the uniform bound above is enough to prove
that the trajectory (or sequence, if discretized) above has the Cauchy
property, hence converges in norm. Recall that a sequence
$\set{x_n}_{n=1}^{\infty}$ is Cauchy if $\forall \varepsilon > 0$, $\exists N$
such that $\abs{x_m - x_n} < \varepsilon, \; \forall m, n > N$.
