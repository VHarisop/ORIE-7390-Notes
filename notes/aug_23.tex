\section{Introduction}

\subsection{Recap}
We briefly review some fundamentals from convex analysis. Recall that, for
\[
	f : \Hbb \to \Rbb \cup \set{+\infty},
\]
where $f$ is convex and $\Hbb$ is a Hilbert space (e.g. $\Rbb^n$), we write
\[
	y \in \partial f(x) \Leftrightarrow x \in \argmin f(\cdot) - \ip{y, \cdot}.
\]
The subgradient $\partial f(x)$ is closed and convex, so it has a unique
shortest element when it is nonempty. This is because the shortest element is
given by the following convex program:
\[
	\inf_{z \in \partial f(x)} \norm{z}^2.
\]
$\norm{~}^2$ is a strongly convex function which is minimized over a closed,
convex set, hence it admits a unique minimizer. Let us call this minimizer
$\partial^o f(x)$.

It can also be shown that the direction of steepest descent is given by
$-\partial^o f(x)$. Hence, for minimization, it is natural to follow the
trajectory given by
\begin{align}
	\begin{cases}
		\dot{X}(t) = -\partial^o f(X(t)), & t \geq 0 \\
		X(0) = x_0. &
	\end{cases}
	\label{eq:descent_trajectory}
\end{align}
This raises a fundamental question:

\paragraph{Question 1: Can we really follow this trajectory?}

The following Theorem provides an affirmative answer:

\begin{ctheorem}{\cite{Brezis73}}{brezis73}
	There exists a trajectory defined by~\cref{eq:descent_trajectory}.
	Moreover, by~\cite{Bruck75}, if that trajectory is bounded, it converges
	weakly to a minimizer of $f$.
\end{ctheorem}

However, in order to be tractable, we would also like to control the length of
those trajectories.

\paragraph{Question 2: How do we control them?}

\subsection{History}
For smooth $f: \Rbb^n \to \Rbb$ (nonconvex):
\begin{itemize}
	\item gradient descent trajectory:
		\[
			\begin{cases}
				\dot{X}(t) = -\grad f(X(t)), & t \geq 0 \\
				X(0) = x_0 \\
			\end{cases}
		\]
		might not have finite length, even if it stays in bounded regions. The
		canonical counterexample involves a spiral ridge or a so-called
		``mexican hat''.
	\item if $f$ is analytic, we have a positive result:
		\begin{ctheorem}{{\L}ojasiewicz, 1984}{loja84}
			When $f$ is real-analytic (i.e. admits a power series expansion),
			bounded gradient descent trajectories have finite length.
		\end{ctheorem}
\end{itemize}
A key tool in the Theorem presented above is the {\L}ojasiewicz inequality,
presented below:
\begin{ctheorem}{\cite{Loja59}}{loja59}
	For real-analytic $f: \Rbb^n \mapsto \Rbb$, around a local minimizer
	$\hat{x}$, it holds that
	\[
		\norm{\grad f(\cdot)} \geq c \abs{f(\cdot) - f(\hat{x})}, \;
		c > 0, \; 0 \leq \mu < 1.
	\]
\end{ctheorem}
An example is $f(x) = x^2$, which satisfies the {\L}ojasiewicz inequality
around $0$ with exponent $\mu = 1/2$.

\subsection{Slope, Sharpness and the KL Property}
It can be shown that the subgradient descent trajectory always attains the
maximum instantaneous rate of decrease. We call this quantity the
\textit{slope}:
\begin{cdefinition}{Slope}{slope}
	We define the slope to be the quantity
	\[
		\abs{\grad f}(x) := \limsup_{z \to x} \frac{f(x) - f(z)}{\norm{x - z}}.
	\]
\end{cdefinition}
Based on the slope, we can characterize a class of functions that are
\textit{sharp} around their minima:
\begin{cdefinition}{Sharpness}{sharpness_informal}
	We call a function $f: \Rbb^n \to \Rbb$ \textbf{sharp} around a minimizer
	$x^{\star}$ if $\abs{\grad f}(x) > \epsilon > 0, \; \forall x \neq
	x^{\star}$ and $x$ close to $x^{\star}$.
\end{cdefinition}
Equipped with the notion of sharpness, it is quite easy to show that
trajectories have finite length. We can often reparameterize the functions of
interest to achieve this; essentially, we are looking for
\begin{equation}
	\phi : \Rbb_+ \mapsto \Rbb \Rightarrow
	\abs{\grad(\phi \circ f)} \geq 1, \; \text{locally}
	\label{eq:kl_ineq_worded}
\end{equation}
\cref{eq:kl_ineq_worded} is called the \textbf{K-L inequality}.

\textbf{Fact}: the length of a trajectory from time $s$ to $t$ satisfies
\[
	X[s, t] \leq \phi(f(X(s))) - \phi(f(X(t))), \; \forall 0 \leq s \leq t
\]
and then $X$ converges (in norm) to a minimizer.


\subsection{Algorithms}
The K-L inequality~\eqref{eq:kl_ineq_worded} is also enough to prove
convergence (and control the rate of convergence) of subgradient descent
\textit{sequences}:
\[
	\exists \alpha, \beta > 0 : \;
	\begin{cases}
		f(x) - f(x_{k+1}) \geq \alpha \norm{x_k - x_{k+1}}^2,  \\
		\abs{\grad f}(x_{k+1}) \leq \beta \norm{x_k - x_{k+1}}.
	\end{cases}
\]
Sharpness gives us the following statement:
\[
	\mathrm{dist}(x, \argmin(f)) \leq \phi(f(x) - \min f)
\]

\paragraph{Question 3: When does the KL property hold?}
The KL property holds for very general classes of functions. For example, it
holds when $f$ is semialgebraic, i.e. a function whose graph is the
intersection of a set of polynomial inequalities.
