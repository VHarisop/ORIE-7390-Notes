\section{Some Convex Analysis}

\subsection{Fundamentals}
The setup: $\Hbb$ is a real Hilbert space (e.g. $\Rbb^n$ for us, but could be
infinite-dimensional).
We denote $\overline{\Rbb} = \Rbb \cup \set{\pm \infty}$. For $f: \Hbb \to
\overline{\Rbb}$, we write
\begin{equation}
    \epi f = \set{(x, r) \in \Hbb \times \Rbb \mmid r \geq f(x)}
    \label{eq:epif}
\end{equation}

It holds that $f$ convex $\Leftrightarrow$ $\epi f$ convex, which is a simple
consequence of convexity. Additionally, $f$ closed implies that $\epi f$ is
closed, which is equivalent to $f$ being lower semicontinuous.

We call $f$ \textit{proper} if it is never $-\infty$, and not always $+\infty$.
Moreover, we define
\begin{equation}
    \dom f = \set{x \in \Hbb \mmid f(x) < \infty} \label{eq:domf}
\end{equation}

\begin{cdefinition}{Subgradient}{subgradient}
    For a convex function $f$, we write $y \in \partial f(x)$ meaning that
    \[
        f(z) \geq f(x) + \ip{y, z - x}, \; \forall z.
    \]
    The set $\partial f(x)$ is always a closed convex set. If it is nonempty,
    it has a \textit{unique} shortest element $\partial^{\circ} f(x)$.
\end{cdefinition}


\begin{ctheorem}{Continuity of convex functions}{conv-cont}
    For a proper convex $f: \Hbb \to \overline{\Rbb}$, at a point $x \in
    \intr{\dom f}$, we know that $f$ is locally Lipschitz on a neighbourhood
    around $x$ with constant $L$. Moreover, its subdifferential satisfies
    \[
        \partial f(x) \neq \emptyset, \quad \norm{\partial f(x)} \leq L.
    \]
\end{ctheorem}
\begin{proof}
    Via the separating hyperplane theorem (Hahn-Banach).
\end{proof}

\subsection{The Fenchel conjugate}
Recall that, for any $f: \Hbb \to \overline{\Rbb}$, we define its
\textbf{conjugate} $f^*$ by
\begin{equation}
    f^*(y) = \sup_{x} \set{\ip{x, y} - f(x)}
    \label{eq:fenchel_conjugate}
\end{equation}

Additionally, from the above definition, we deduce the so-called
\textit{Fenchel-Young inequality}:
\[
    f(x) + f^*(y) \geq \ip{x, y}, \; \forall x, y.
\]

\begin{ctheorem}{Biconjugate}{biconjugate}
    For closed, convex, proper $f: \Hbb \to \overline{\Rbb}$, it holds that
    $f^{**} = f$.
\end{ctheorem}

We briefly give some examples:
\begin{itemize}
\item Consider $\cC \subseteq \Hbb$. We define its \textit{indicator function}
    $\delta_{\cC}(x) = \begin{cases}
        0, & x \in \cC \\
        +\infty, & x \notin \cC
    \end{cases}$. Its conjugate is the \textit{support function} of $\cC$:
    \[
        \delta_{\cC}^*(x) = \sup{z} \set{\ip{x, z} - \delta_{\cC}(z)}
        = \sup_{z \in \cC} \ip{x, z},
    \]
    where the simplification above occurs since $-\delta_{\cC}(z)$ is $-\infty$
    for $z \notin \cC$, which is a case that we disregard when computing the
    supremum, otherwise $0$.
\item For an indicator function $\delta_{\cC}$, its subgradient at $x$ is the
    so-called \textit{normal cone} at $x$, denoted by
    \[
        N_{\cC}(x) := \partial \delta_{\cC}(x).
    \]
    Its conjugate is defined as
    \[
        \partial \delta_{\cC}^*(y) := \argmax_{x \in \cC} \ip{x, y}.
    \]
\end{itemize}

\begin{exercise}{Conjugates}{indicator-conjugates}
    Prove the forms of the conjugates in the examples above.
\end{exercise}

The next proposition uncovers a relationship between the subgradients of the
original function and its Fenchel conjugate.
\begin{cproposition}{Subgradients of conjugates}{conj-subgrad}
    For a closed, convex, proper $f$, we have $y \in \partial f(x)$ exactly when
    $x \in \partial f^*(y)$. This is precisely when equality holds in the
    Fenchel-Young inequality.
\end{cproposition}
\begin{proof}
    The proof is left as an exercise.
\end{proof}

\begin{cdefinition}{Positive Homogeneity}{pos-homogen}
    We call a function $f: \Hbb \mapsto \overline{\Rbb}$ positively homogeneous
    if
    \[
        f(tx) = tf(x), \; \forall t \geq 0, \; \forall x,
    \]
    and, additionally, $f(0) = 0$.
\end{cdefinition}
The next proposition relates the positive homogeneity of support functions with
the set $\cC$ having certain desirable properties:
\begin{cproposition}{~}{support-fun-convex}
    The support function $f = \delta_{\cC}^*$ is proper, closed, convex and
    positively homogeneous when $\cC$ is a nonempty, closed, convex set that
    satisfies $\cC \neq \Hbb$. The converse is also true.
\end{cproposition}
\begin{proof}
    We prove each direction separately:
    \begin{itemize}
    \item $\Rightarrow$: the proof is immediate. \todo{Write down details}
    \item $\Leftarrow$: Let $\cC = \partial f(0)$. Given a $y \in \partial
    f(0)$, it holds that
    \[
        \ip{x, y} + \cancelto{0}{f(0)} \leq f(x), \; \forall x.
    \]
    Then, by the definition of the Fenchel conjugate:
    \[
        f^*(y) = \sup_{x} \set{\underbrace{\ip{x, y} - f(x)}_{\leq 0}}
        \leq 0,
    \]
    and equality is satisfied for $x = 0$. If $y \notin \partial f(0)$, then
    $\exists \bar{x}$ such that $\ip{\bar{x}, y} > f(\bar{x})$ (otherwise $y$
    would satisfy the subgradient inequality!). This implies
    \[
        f^*(y) \geq \sup_{t \geq 0} \set{\ip{t\bar{x}, y} - f(t \bar{x})}
        = \sup_{t \geq 0} t \set{\ip{\bar{x}, y} - f(\bar{x})} = +\infty
    \]
    This means that $f^* = \delta_{\cC}$ (\textbf{why?}) hence $f = f^{**} =
    \delta_{\cC}^*$.
    \end{itemize}
\end{proof}
