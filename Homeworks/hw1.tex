\documentclass[10pt]{article}

% Required packages
\usepackage{bm,bbm}
\usepackage{amsmath,amssymb,amsthm,cancel}
\usepackage{algorithm, algpseudocode}
\usepackage{minted, caption}



% Color references
\usepackage[
    colorlinks=true, citecolor=green, linkcolor=blue]{hyperref}

\newcommand{\homework}[2]{
	\noindent
    \begin{center}
    	\framebox{
        	\vbox{
            	% Course Title and Date
            	\hbox to \hsize { \textsc{ORIE 7390 - Special Topics in
				Mathematical Programming} \hfill \textsc{#1} }
                \vspace{4mm}
                % Title of handout/homework
                \hbox to \hsize { {\Large \hfill \textsc{#2} \hfill} }
                \vspace{1mm}
            }
        }
    \end{center}
}

\newenvironment{alglist}{\begin{list}{}{\setlength{\leftmargin}{1.5cm}
\setlength{\rightmargin}{0cm}\setlength{\itemsep}{1ex}\setlength{\parsep}{1ex}}}{\end{list}}

\newcommand{\problem}[3]
{\fbox{\parbox{6in}{{\bf #1}\begin{itemize}\item{\bf Input:} {#2} \item{\bf Goal:} {#3}\end{itemize}}}}

\usepackage{latex-macros}
\usepackage{todonotes}
\usepackage{tikz,pgfplots}
\usepackage{algorithm, algpseudocode}
\usepackage[margin=1in]{geometry}
\usetikzlibrary{shapes.geometric}
%\usepackage{parskip}
\usepackage[capitalize]{cleveref}
\usepackage{exercise}

\usepackage{todonotes}

\newcommand{\regdiff}{\hat{\partial}}
\newcommand{\bd}[1]{\mathrm{bd}\left( #1 \right)}

\begin{document}

\allowdisplaybreaks
\everymath{\displaystyle}

\homework{Vasileios Charisopoulos}{Homework 1}

\begin{Exercise}
	Consider $f(u, v) = \abs{u} + v^2$, so $\dom f = \Rbb^2$.

	\ExePart

		Let us simplify the notation by dropping $\lambda$ from $x_k^{\lambda}$
		for now, until we determine the solution of the proximal map.
		Consider $x^* = \argmin_{x} f(x) + \frac{\lambda}{2} \norm{x - x_k}^2$.
		Since $f$ is a convex function, the proximal map is well defined and,
		additionally, based on the $f$ having full domain, we can conclude that
		\[
			\partial \left(f(x) + \frac{\lambda}{2} \norm{x - x_k}^2 \right)
			= \partial f(x) + \partial \left( \frac{\lambda}{2} \norm{x -
			x_k}^2 \right).
		\]
		Therefore we simply have to write the first-order optimality conditions
		for $x^*$, which means that
		\begin{align*}
			0 &\in \partial f(x^*) + \lambda ( x^* - x_k ) \Rightarrow
			0 \in \begin{pmatrix} \partial \abs{x^*_1} \\ 2 x^*_2 \end{pmatrix}
				+ \lambda \begin{pmatrix} x^*_1 - x_k^{(1)} \\ x^*_2 - x_k^{(2)}
				\end{pmatrix},
		\end{align*}
		which we arrived at observing that the function minimized is separable.
		For the smooth part, trivial algebraic manipulations lead to $x_2^* =
		\frac{\lambda}{2 + \lambda} x_k^{(2)}$. For the nonsmooth part, we
		know from ORIE 6328 that the solution is given by the
		\textit{soft thresholding} operator. Nevertheless, we repeat the
		derivation to convince the reader. Consider the following cases:
		\begin{enumerate}
			\item $x^*_1 > 0$: in that case $\partial \abs{x_1}^* = 1$ and
				$x_k^{(1)} = x_1 + \frac{1}{\lambda}$.
			\item $x^*_1 < 0$: like before, we have $\partial \abs{x_1}^* = -1$
				leading to $x_k^{(1)} = x_1 - \frac{1}{\lambda}$.
		\end{enumerate}
		The two cases above imply that $\sign(x_k^{(1)}) = \sign(x_1^*)$ when
		$x_1^* \neq 0$. Now, consider the case where it is equal to $0$. We
		know from convex analysis that $\partial \abs{x} = [-1, 1]$ when
		$x = 0$, so $x_k^{(1)}$ must be $\in \frac{[-1, 1]}{\lambda}$. Gathering all the
		cases above (and keeping in mind that $\lambda > 0$) gives us
		\begin{align*}
			x_1^* &= \sign(x_k^{(1)}) \max\left(\abs{x_k^{(1)}} -
				\frac{1}{\lambda}, 0\right).
		\end{align*}
		Therefore, we have
		\[
			x_{k+1}^{\lambda} = \begin{pmatrix}
				\sign(x_k^{(1)}) \max\left( \abs{x_k^{(1)}} -
				\frac{1}{\lambda}, 0 \right) \\
				\frac{\lambda}{2 + \lambda} x_k^{(2)}
			\end{pmatrix}, \; \forall \lambda > 0.
		\]
\end{Exercise}

\begin{Exercise}
	Consider $f: \Rbb^n \to \Rbb$, continuously differentiable. This means that
	$f$ satisfies the following at any point $x$:
	\begin{align}
		f(x + d) &= f(x) + \ip{\grad f(x), d} + o(d), \;
		\lim_{d \to 0} \frac{o(d)}{\norm{d}} = 0.
		\label{eq:diff-defn}
	\end{align}
	First, consider $x$ being a minimizer. Then, we have defined the slope to
	be $0$ by convention, which agrees with the first-order condition $\grad
	f(x) = 0$ in unconstrained optimization. The nontrivial case has $x$ not
	being a minimizer.

	Let us work with the definition of the slope. We write
	\begin{align*}
		\abs{\grad f}(x) &= \limsup_{z \to x} \frac{f(x) - f(z)}{\norm{x - z}}
		= \lim_{\delta \dto 0} \sup_{\norm{z - x} \leq \delta} \frac{f(x) -
		f(z)}{\norm{x - z}} \\
			&= \lim_{\delta \dto 0} \sup_{\norm{d} \leq \delta}
			\frac{f(x) - f(x + d)}{\norm{d}} =
				\lim_{\delta \dto 0} \sup_{\norm{d} \leq \delta}
			\frac{f(x) - f(x) - \ip{\grad f(x), d} + o(d)}{\norm{d}} \\
			&\overset{(\text{Cauchy-Schwarz})}{\leq} \limsup_{d \to 0} \left(
				\frac{\norm{\grad f(x)} \norm{d}}{\norm{d}} +
			  	\frac{o(d)}{\norm{d}} \right) = \norm{\grad f(x)},
	\end{align*}
	where we used the fact that $\frac{o(d)}{\norm{d}} = 0$ as $d \to 0$.
	It is left to show that $\abs{\grad f}(x) \geq \norm{\grad f(x)}$, and the
	proof will be complete. To that end, observe that since $z \to x$ in the
	limit superior of $\abs{\grad f}$'s definition, we must have for any
	$d \in \ball_2$:
	\[
		\abs{\grad f}(x) \geq \lim_{t \to 0} \frac{f(x) - f(x + td)}{t},
	\]
	which is simply the mathematical way to state the fact that approaching $x$
	by arbitrary trajectories can only give us a bigger $\sup$ than by
	approaching it from a single direction. Then, setting $d = -\frac{\grad
	f(x)}{\norm{\grad f(x)}}$ and replacing $f(x + td)$ by~\cref{eq:diff-defn}
	gives us:
	\begin{align*}
		\abs{\grad f}(x) &\geq
		\lim_{t \dto 0} \frac{f(x) - f(x) + t \norm{\grad f(x)} + o(t)}{t}
			= \norm{\grad f(x)},
	\end{align*}
	which completes the claim. Hence $\abs{\grad f}(x) = \norm{\grad f(x)}$
	when $f$ is $C^1$.
\end{Exercise}

\end{document}
