\documentclass[10pt]{beamer}

\usepackage{default}
\usepackage{xltxtra}
\usepackage{fontspec}
\usepackage{latex-macros}
\usepackage{cancel}
%\usefonttheme[onlymath]{serif}
\usefonttheme{professionalfonts}

\usepackage{cleveref}
\usepackage{svg}
\usepackage{pgf}
\usepackage{tikzducks, tikz}
\usetikzlibrary{arrows}
\usepackage{mathrsfs}
\usepackage[sort]{natbib}

\mode<presentation>
{
    \usetheme
    [navbar=true,colorblocks=true,pagenumbers=true]{Cornell}
}

\usepackage[skins]{tcolorbox}
\newcommand{\xstar}{x^{\star}}
\newcommand{\eps}{\varepsilon}

\allowdisplaybreaks

\begin{document}

\title{On the convergence of the proximal algorithm for nonsmooth functions
involving analytic features}
\author{Hedy Attouch \& J{\'e}{\^o}me Bolte}
\frame{\titlepage}
\frame{\tableofcontents}

%% INTRODUCTION: Problem motivation, difficulties and prior art
\section{Introduction}

\begin{frame}{Introduction}
Recall the definition of the \textit{proximal point} method:
\begin{equation}
	x_{k+1} \in \mathrm{prox}_{\lambda_k f}
	:= \argmin_{u \in \Rbb^d} \set{
	f(u) + \frac{1}{2 \lambda_k} \norm{u - x_k}^2 }, \;
	x_0 \in \Rbb^d
	\label{eq:proximal-point-method}
\end{equation}
Our setting:
\begin{itemize}
	\item $f: \Rbb^d \to \overline{\Rbb}$: proper, lsc
	\item $\lambda_k > 0, \; \forall k$
	\item $\inf_{\Rbb^d} f > -\infty$
\end{itemize}
\end{frame}

\section{Introduction \& Background}

\begin{frame}{Introduction}
	Two main questions:
	\begin{enumerate}
		\item<1-> Does the sequence defined by~\cref{eq:proximal-point-method}
			converge?
		\item<2-> Is there a generic way to say \textit{how fast} it does?
	\end{enumerate}

	\visible<3->{
		Tackle both using \textbf{{\L}ojasiewicz exponents}, which appear in a
		special case of the Kurdyka-{\L}ojasiewicz inequality.
	}
\end{frame}

\begin{frame}{Background}
	Recall the following definitions and facts:
	\begin{enumerate}
		\item \textbf{Critical points} of $f$: denote
			\begin{align*}
				\mathrm{crit}(f) := \set{x \in \dom f \mmid 0 \in \partial
				f(x)},
			\end{align*}
			where $\partial f$ denotes the \textbf{limiting} subdifferential.
		\item For a proper lsc $f$ with $\inf f > -\infty$ it holds that
			\[
				x_{k+1} \in \mathrm{prox}_{\lambda_k f}
				\Rightarrow \frac{x_k - x_{k+1}}{\lambda_k} \in \partial f(x).
			\]
	\end{enumerate}
\end{frame}

\section{Main Results}

\begin{frame}{{\L}ojasiewicz exponents}
	A slightly specialized version of the familiar KL inequality:
	\begin{definition} \label{defn:loj-exp}
		A function $f$ sat. the {\L}ojasiewicz property around $\hat{x} \in
		\mathrm{crit}(f)$ if $\exists C, \eps > 0, \theta \in [0, 1)$ such that
		\[
			C \norm{v} \abs{f(x) - f(\hat{x})}^{1 - \theta} \geq 1, \;
			\forall v \in \partial f(x), \; \forall x \in \hat{x} + \eps
			\mathbb{B}.
		\]
	\end{definition}
	Note: the {\L}ojasiewicz property is a special case of the KL inequality
	with desingularizer $\varphi(s) = C s^{1 - \theta}$.
\end{frame}

\begin{frame}{Main takeaway}
	Two positive answers to our earlier questions:
	\begin{enumerate}
		\item Bounded sequences $\set{x_k}$ converge to critical points of $f$
		\item Rate of convergence is controlled by {\L}ojasiewicz exponent
	\end{enumerate}
\end{frame}

\section{

\begin{frame}[allowframebreaks]{References}
\bibliographystyle{plain}
\bibliography{references}
\end{frame}

\end{document}
