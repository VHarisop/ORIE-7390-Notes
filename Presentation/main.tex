\documentclass[usenames, dvipsnames, 10pt]{beamer}


\usepackage{default}
\usepackage{xltxtra}
\usepackage{fontspec}
\usepackage{latex-macros}
\usepackage{cancel}
\usetheme[block=fill, background=light, titleformat=smallcaps]{metropolis}
\usefonttheme{professionalfonts}
\usepackage{cleveref}
\usepackage[sort]{natbib}

% \setsansfont{CMU Sans Serif Demi Condensed}

\newcommand{\xstar}{x^{\star}}
\newcommand{\eps}{\varepsilon}

\theoremstyle{definition}
\newtheorem{proposition}{Proposition}

\begin{document}

\title{On the convergence of the proximal algorithm for nonsmooth functions
involving analytic features}
\subtitle{Hedy Attouch, J{\'e}r{\^o}me Bolte}
\author{Presenter: Vasilis Charisopoulos}
\frame{\titlepage}
\frame{\tableofcontents}

\allowdisplaybreaks

%% INTRODUCTION: Problem motivation, difficulties and prior art
\section{Introduction}

\begin{frame}{Introduction}
Recall the definition of the \textit{proximal point} method:
\begin{equation}
    \color{MidnightBlue}
	x_{k+1} \in \mathrm{prox}_{\lambda_k f}
	:= \argmin_{u \in \Rbb^d} \set{
	f(u) + \frac{1}{2 \lambda_k} \norm{u - x_k}^2 }, \;
	x_0 \in \Rbb^d
	\label{eq:proximal-point-method}
\end{equation}
Our setting:
\begin{itemize}
	\item $f: \Rbb^d \to \overline{\Rbb}$: proper, lsc
	\item $0 < \lambda_- < \lambda_k < \lambda_+, \; \forall k$
	\item $\inf_{\Rbb^d} f > -\infty$
\end{itemize}
\end{frame}

\begin{frame}{Introduction}
	Two main questions:
	\begin{enumerate}
		\item Does the sequence defined by~\cref{eq:proximal-point-method}
			converge?
		\item Is there a generic way to say \textit{how fast} it does?
	\end{enumerate}

	\visible<2->{
		Tackle the latter using \textbf{{\L}ojasiewicz exponents}, which
		specialize the familiar Kurdyka-{\L}ojasiewicz property.
	}
\end{frame}

\section{Background}

\begin{frame}{Refresher}
	\begin{enumerate}
		\item \textbf{Critical points} of $f$: denote
			\begin{align*}
            {\color{MidnightBlue}
				\mathrm{crit}(f) := \set{x \in \dom f \mmid 0 \in \partial
				f(x)}},
			\end{align*}
			where $\partial f$ denotes the \textbf{limiting} subdifferential.
		\item For a proper lsc $f$ with $\inf f > -\infty$ it holds that
        \citep[Chapter 10]{RockWett98}:
			\[
                \color{MidnightBlue}
				x_{k+1} \in \mathrm{prox}_{\lambda_k f}
				\Leftrightarrow \frac{x_k - x_{k+1}}{\lambda_k} \in \partial
				f(x).
			\]
	\end{enumerate}
\end{frame}

\begin{frame}{{\L}ojasiewicz property}
	A slightly specialized version of the familiar KL property:
	\begin{definition} \label{defn:loj-exp}
		A function $f$ sat. the {\L}ojasiewicz property around $\hat{x} \in
		\mathrm{crit}(f)$ if $\exists C, \eps > 0, \theta \in [0, 1)$ such that
		\[ \color{MidnightBlue} \boxed{
			C \norm{v} \abs{f(x) - f(\hat{x})}^{1 - \theta} \geq 1, \;
			\forall v \in \partial f(x), \; \forall x \in \hat{x} + \eps
			\mathbf{B}}
		\]
	\end{definition}
	\textbf{Note}: the {\L}ojasiewicz property is a special case of the KL
	inequality with desingularizer $\varphi(s) = C s^{1 - \theta}$.
\end{frame}

\begin{frame}{Main takeaway}
	Two positive answers to our earlier questions:
	\begin{enumerate}
		\item Bounded sequences $\set{x_k}$ converge to critical points of $f$
		\item Rate of convergence is controlled by {\L}ojasiewicz exponent
	\end{enumerate}
\end{frame}

\section{Convergence: Preliminaries}

\begin{frame}{Proximal Point Sequences}
    \begin{proposition}[Proximal point sequences] \vspace{1pt}
    Consider $\set{x^k}_{k \in \Nbb}$ generated by the proximal point iteration,
    and $\cX(x^0)$ its limit points starting from $x^0$. Then:
    \begin{enumerate}
    \item \label{P1} $\set{f(x^k)}$ is nonincreasing,
    \item \label{P2} $\sum_k \norm{x^k - x^{k+1}}^2$ is finite,
    \item \label{P3}
    if $\set{x^k}$ bounded, then $\cX(x^0)$ compact, connected, and
    $\mathrm{dist}(\cX, x^k) \to 0$
    \item \label{P4} if $f$ continuous, $\cX(x^0) \subset \mathrm{crit}(f)$ and
    $f$ is
    constant on $\cX(x^0)$
    \end{enumerate}
    \end{proposition}
\end{frame}
\begin{frame}{Proximal point sequences}
    \begin{proof}[Proof sketch] \vspace{1pt}
        Item~\eqref{P1} follows from defn. of proximal point, and~\eqref{P2}
        follows after summing up. Item~\eqref{P3} was homework. For
        item~\eqref{P4}, denote $\xbar$ a limit point and
        \[
            (2) \to \lim_{k \to \infty} \norm{x^k - x^{k+1}}^2 = 0
            \Rightarrow \lim_{k \to \infty}
            \frac{1}{\lambda_k} \norm{g^k}^2 = 0, \;
        \]
        for $g^k \in \partial f(x)$,
        so $\cX(x^0) \subset \mathrm{crit}(f)$ since
        $(\xbar, 0) \in \gph \partial f$ by continuity of $f$.
    \end{proof}
\end{frame}

\begin{frame}{Consequences of L-property}
    If $K$ compact, $f$ satisfies L-property with exponent $\theta$:
    \[
        \color{MidnightBlue}
        \forall x \in K + \epsilon \mathbf{B}:
        \abs{f(x) - f(\xhat)}^{\theta} \leq C \mathrm{dist}(0, \partial f(x)).
    \]
    Proved by topological definition of compactness and L-property.

        $\implies$ $f$ is \textbf{constant} on any connected subset of its
        critical points.
\end{frame}

\begin{frame}{Functions satisfying L-property}
    \begin{itemize}
    \item Real-analytic~\citep{Loja63} \& Sub-analytic
    functions~\citep{BolDanLew07}
    \end{itemize}
\end{frame}

\section{Main Results}
\subsection{Convergence}
\begin{frame}{Convergence \& an error bound}
    \begin{block}{\cite[Theorem 1]{AttBol09}} \vspace{1pt}
        Assume $f$ as before and a sequence $\set{x^k}$ from the proximal point
        iteration. If sequence is bounded, then
        \[	\color{MidnightBlue}
            \sum_k \norm{x^{k+1} - x^k} < \infty,
        \]
        and the whole sequence converges to some $\xbar \in \mathrm{crit}(f)$.
    \end{block}
\end{frame}
\begin{frame}{Proof sketch}
    As usual, assume wlog $\inf_k f(x^k) = 0$.
    Note: $f(x^k) \geq f(x^{k+1}) + (\lambda_k)^{-1} \norm{x^k - x^{k+1}}^2$
    from proximal point iteration. Use subgradient inequality on $x \mapsto
    -x^{1-\theta}$ to obtain
    \begin{align*}
        \color{MidnightBlue}
        {f(x^k)}^{1-\theta} - {f(x^{k+1})}^{1-\theta} \geq (1 - \theta)
        {f(x^k)}^{-\theta}
            \frac{\norm{x^{k+1} - x^k}^2}{2 \lambda_k},
    \end{align*}
    and the L-property for a large enough $k$ to obtain
    \begin{align*}
        \color{MidnightBlue}
        \frac{\norm{x^{k+1} - x^k}^2}{\norm{x^k - x^{k-1}}}
        \leq M \left(f(x^k)^{1-\theta} - f(x^{k+1})^{1-\theta}\right) \\
        \color{MidnightBlue}
        \overset{(\text{algebra...})}{\Rightarrow}
        \norm{x^{k+1} - x^k} \leq r \norm{x^k - x^{k-1}}
        + \frac{M}{r} \left(f(x^k)^{1-\theta} - f(x^{k+1})^{1-\theta}\right),
    \end{align*}
    for some $r \in (0, 1)$. ``Telescope'' both sides to obtain convergence,
    Cauchy property.
\end{frame}

\subsection{Rate of convergence}

\begin{frame}{Deducing the rate of convergence}
    \begin{block}{\cite[Theorem 2]{AttBol09}}
        Assume L-property holds for $f$ around a limit point $x^{\star}$ of
        a bounded sequence $\set{x^k}$. Then:
        \begin{itemize}
        \item $\theta = 0$: converge in finite number of steps
        \item $\theta \in (0, 1/2]$: converge as $\norm{x^k - x^{\star}}
        = \cO(Q^k), \; Q < 1$
        \item $\theta \in (1/2, 1)$: converge as
        $\norm{x^k - x^{\star}} = \cO(k^{-\frac{1-\theta}{2\theta - 1}})$
        \end{itemize}
    \end{block}
\end{frame}

\begin{frame}[allowframebreaks]{References}
\bibliographystyle{humannat}
%\bibliographystyle{plain}
\bibliography{references}
\end{frame}

\end{document}
